% Options for packages loaded elsewhere
\PassOptionsToPackage{unicode}{hyperref}
\PassOptionsToPackage{hyphens}{url}
%
\documentclass[
]{article}
\usepackage{amsmath,amssymb}
\usepackage{iftex}
\ifPDFTeX
  \usepackage[T1]{fontenc}
  \usepackage[utf8]{inputenc}
  \usepackage{textcomp} % provide euro and other symbols
\else % if luatex or xetex
  \usepackage{unicode-math} % this also loads fontspec
  \defaultfontfeatures{Scale=MatchLowercase}
  \defaultfontfeatures[\rmfamily]{Ligatures=TeX,Scale=1}
\fi
\usepackage{lmodern}
\ifPDFTeX\else
  % xetex/luatex font selection
\fi
% Use upquote if available, for straight quotes in verbatim environments
\IfFileExists{upquote.sty}{\usepackage{upquote}}{}
\IfFileExists{microtype.sty}{% use microtype if available
  \usepackage[]{microtype}
  \UseMicrotypeSet[protrusion]{basicmath} % disable protrusion for tt fonts
}{}
\makeatletter
\@ifundefined{KOMAClassName}{% if non-KOMA class
  \IfFileExists{parskip.sty}{%
    \usepackage{parskip}
  }{% else
    \setlength{\parindent}{0pt}
    \setlength{\parskip}{6pt plus 2pt minus 1pt}}
}{% if KOMA class
  \KOMAoptions{parskip=half}}
\makeatother
\usepackage{xcolor}
\usepackage[margin=1in]{geometry}
\usepackage{graphicx}
\makeatletter
\def\maxwidth{\ifdim\Gin@nat@width>\linewidth\linewidth\else\Gin@nat@width\fi}
\def\maxheight{\ifdim\Gin@nat@height>\textheight\textheight\else\Gin@nat@height\fi}
\makeatother
% Scale images if necessary, so that they will not overflow the page
% margins by default, and it is still possible to overwrite the defaults
% using explicit options in \includegraphics[width, height, ...]{}
\setkeys{Gin}{width=\maxwidth,height=\maxheight,keepaspectratio}
% Set default figure placement to htbp
\makeatletter
\def\fps@figure{htbp}
\makeatother
\usepackage{svg}
\setlength{\emergencystretch}{3em} % prevent overfull lines
\providecommand{\tightlist}{%
  \setlength{\itemsep}{0pt}\setlength{\parskip}{0pt}}
\setcounter{secnumdepth}{-\maxdimen} % remove section numbering
\newlength{\cslhangindent}
\setlength{\cslhangindent}{1.5em}
\newlength{\csllabelwidth}
\setlength{\csllabelwidth}{3em}
\newlength{\cslentryspacingunit} % times entry-spacing
\setlength{\cslentryspacingunit}{\parskip}
\newenvironment{CSLReferences}[2] % #1 hanging-ident, #2 entry spacing
 {% don't indent paragraphs
  \setlength{\parindent}{0pt}
  % turn on hanging indent if param 1 is 1
  \ifodd #1
  \let\oldpar\par
  \def\par{\hangindent=\cslhangindent\oldpar}
  \fi
  % set entry spacing
  \setlength{\parskip}{#2\cslentryspacingunit}
 }%
 {}
\usepackage{calc}
\newcommand{\CSLBlock}[1]{#1\hfill\break}
\newcommand{\CSLLeftMargin}[1]{\parbox[t]{\csllabelwidth}{#1}}
\newcommand{\CSLRightInline}[1]{\parbox[t]{\linewidth - \csllabelwidth}{#1}\break}
\newcommand{\CSLIndent}[1]{\hspace{\cslhangindent}#1}
\ifLuaTeX
  \usepackage{selnolig}  % disable illegal ligatures
\fi
\IfFileExists{bookmark.sty}{\usepackage{bookmark}}{\usepackage{hyperref}}
\IfFileExists{xurl.sty}{\usepackage{xurl}}{} % add URL line breaks if available
\urlstyle{same}
\hypersetup{
  pdftitle={Migratory decisions in birds with different migration strategies during spring},
  hidelinks,
  pdfcreator={LaTeX via pandoc}}

\title{\textbf{Migratory decisions in birds with different migration
strategies during spring}}
\author{true \and true \and true \and true}
\date{}

\begin{document}
\maketitle

\hypertarget{introduction}{%
\section{Introduction}\label{introduction}}

Bird migration is defined by a sequential series of \textbf{trade-off
decisions}, including departure, routing, and landing decisions. In
combination, these three aspects shape the spatio-temporal patterns of
an individual movement, and are thus directly linked to the distance
travelled per time unit and to energy consumption, i.e.~cost of
transport under variable environmental conditions.
(\protect\hyperlink{ref-schmaljohann2022}{Schmaljohann, Eikenaar, and
Sapir 2022}).

Individual migratory decisions during autumn migration likely depend on
\textbf{migration strategy}, i.e.~short- or long-distance migrants, and
birds of both strategies differently react to prevailing environmental
conditions at stopover (\protect\hyperlink{ref-packmor2020}{Packmor et
al. 2020}). However, it remains unclear whether migration strategy
similarly affects the adjustment of migratory decisions during spring,
when early arrivals at the breeding grounds should be mutually
beneficial for individual reproductive fitness.

Objectives

\begin{enumerate}
\def\labelenumi{\arabic{enumi}.}
\tightlist
\item
  Do departure and routing decisions differ between migration strategies
  in spring?
\item
  How many birds cross the German Bight?
\item
  Does migration strategy affect how birds adjust migratory decisions?
\end{enumerate}

\hypertarget{methods}{%
\section{Methods}\label{methods}}

\href{https://motus.org/}{\includegraphics[width=0.9\textwidth,height=\textheight]{figures/motus.png}}

We equipped 289 songbirds of seven species from both migration
strategies with radio tags at coastal stopover sites along the German
North Sea coast during spring and tracked them by means of an automated
receiver network. Once departed, birds could either cross the German
Bight or take a detour along the coast. Using a hierarchical multistate
model, we estimated weather effects on daily migratory decisions,
i.e.~day-to-day departure decisions in concert with routing.

\newpage

\hypertarget{results}{%
\section{Results}\label{results}}

\begin{figure}
\centering
\includesvg[width=0.69\textwidth,height=\textheight]{figures/mean.svg}
\caption{Departure probability was higher in long-distance migrants. The
mean probability for an offshore flight differed between species but not
migration strategies.}
\end{figure}

\begin{enumerate}
\def\labelenumi{\arabic{enumi}.}
\tightlist
\item
  Day-to-day departure probability among species was higher in
  long-distance migrants independently from routing decision.
\item
  We estimated that 56\% (95\% CrI: 47.8--58.2 \%) of all birds crossed
  the German Bight.
\end{enumerate}

\begin{figure}
\centering
\includesvg[width=0.8\textwidth,height=\textheight]{figures/dep_time.svg}
\caption{Short-distance migrants stayed on average 7.8 days (5.2--11.1
days) longer than long-distance migrants at the stopover sites. Within
the day of departure, offshore flights started 5.4\% of night length
(2.1--8.9 \%) earlier and with less variation compared to onshore
flights.}
\end{figure}

\begin{figure}
\centering
\includesvg[width=1\textwidth,height=\textheight]{figures/dep_timing_1.svg}
\caption{Birds more likely departed under westerly winds (easterly winds
in Northern Wheatears) and light southerly winds.}
\end{figure}

The influence of air pressure change and low relative humidity differed
between species but not migration strategy.

Birds more likely departed during times with no precipitation.

\includesvg[width=0.83\textwidth,height=\textheight]{figures/dep_decision.svg}

\begin{enumerate}
\def\labelenumi{\arabic{enumi}.}
\setcounter{enumi}{2}
\tightlist
\item
  We found no consistent differences in reaction norms to prevailing
  environmental conditions between migration strategies.
\end{enumerate}

\hypertarget{conclusion}{%
\section{Conclusion}\label{conclusion}}

Studying proximate mechanisms on individual departure and routing
decisions in concert, our results suggest that migration timing during
spring inherently depends on migration strategy, while individual
weather related adjustments of migratory decisions are similar between
strategies. We therefore suppose that, despite high individual en route
flexibility, selection similarly affects birds of different migration
strategies during spring in favour of early arrivals at the breeding
grounds.

\hypertarget{references}{%
\section{References}\label{references}}

\hypertarget{refs}{}
\begin{CSLReferences}{1}{0}
\leavevmode\vadjust pre{\hypertarget{ref-packmor2020}{}}%
Packmor, Florian, Thomas Klinner, Bradley K Woodworth, Cas Eikenaar, and
Heiko Schmaljohann. 2020. {``Stopover Departure Decisions in Songbirds:
Do Long-Distance Migrants Depart Earlier and More Independently of
Weather Conditions Than Medium-Distance Migrants?''} \emph{Movement
Ecology} 8 (1): 1--14. \url{https://doi.org/10.1186/s40462-020-0193-1}.

\leavevmode\vadjust pre{\hypertarget{ref-schmaljohann2022}{}}%
Schmaljohann, Heiko, Cas Eikenaar, and Nir Sapir. 2022. {``Understanding
the Ecological and Evolutionary Function of Stopover in Migrating
Birds.''} \emph{Biological Reviews}.
\url{https://doi.org/10.1111/brv.12839}.

\end{CSLReferences}

{*}

Map indicating locations of tag deployment (triangles) and receiver
stations where birds arrived after a migratory endurance flight (dots,
size equals to the number of individuals). The histogram on the left
summarises the number of individuals detected per 0.1°. Offshore
detections on Helgoland and FINO3 are given in light colours. Dashed
black line indicates threshold latitude and longitude for flight
categorisation as offshore (to the left) or onshore (to the right)
flight.

\includegraphics[width=0.5\textwidth,height=\textheight]{figures/IfV.png}

\end{document}
